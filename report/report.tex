\documentclass{article}
\RequirePackage[utf8]{inputenc}
\usepackage[portuguese]{babel}
\usepackage{listings}
\usepackage{color}

\definecolor{dkgreen}{rgb}{0,0.6,0}
\definecolor{gray}{rgb}{0.5,0.5,0.5}
\definecolor{mauve}{rgb}{0.58,0,0.82}

\lstset{frame=tb,
  language=Java,
  aboveskip=3mm,
  belowskip=3mm,
  showstringspaces=false,
  columns=flexible,
  basicstyle={\small\ttfamily},
  numbers=none,
  numberstyle=\tiny\color{gray},
  keywordstyle=\color{blue},
  commentstyle=\color{dkgreen},
  stringstyle=\color{mauve},
  breaklines=true,
  breakatwhitespace=true,
  tabsize=3,
  literate=
  {á}{{\'a}}1 {é}{{\'e}}1 {í}{{\'i}}1 {ó}{{\'o}}1 {ú}{{\'u}}1
  {Á}{{\'A}}1 {É}{{\'E}}1 {Í}{{\'I}}1 {Ó}{{\'O}}1 {Ú}{{\'U}}1
  {à}{{\`a}}1 {è}{{\'e}}1 {ì}{{\`i}}1 {ò}{{\`o}}1 {ù}{{\`u}}1
  {À}{{\`A}}1 {È}{{\'E}}1 {Ì}{{\`I}}1 {Ò}{{\`O}}1 {Ù}{{\`U}}1
  {ä}{{\"a}}1 {ë}{{\"e}}1 {ï}{{\"i}}1 {ö}{{\"o}}1 {ü}{{\"u}}1
  {Ä}{{\"A}}1 {Ë}{{\"E}}1 {Ï}{{\"I}}1 {Ö}{{\"O}}1 {Ü}{{\"U}}1
  {â}{{\^a}}1 {ê}{{\^e}}1 {î}{{\^i}}1 {ô}{{\^o}}1 {û}{{\^u}}1
  {Â}{{\^A}}1 {Ê}{{\^E}}1 {Î}{{\^I}}1 {Ô}{{\^O}}1 {Û}{{\^U}}1
  {œ}{{\oe}}1 {Œ}{{\OE}}1 {æ}{{\ae}}1 {Æ}{{\AE}}1 {ß}{{\ss}}1
  {ç}{{\c c}}1 {Ç}{{\c C}}1 {ø}{{\o}}1 {å}{{\r a}}1 {Å}{{\r A}}1
  {€}{{\EUR}}1 {£}{{\pounds}}1
}

\begin{document}

\title{Relatório do primeiro trabalho de PDM}
\author{David Raposo nº33724\and Pedro Pedroso nº32632 \and Ricardo Mata nº33404}

\maketitle

\section{Introdução}
O primeiro trabalho da disciplina de Programação de Dispositivos Móveis visa a criação de uma aplicação que, utilizando os conhecimentos básicos
adquiridos sobre a plataforma Android, permite que o utilizador possa ver alguma informação lectiva sobre as cadeiras do curso de Informática, com base em
informação extraída a partir da API do Thoth \footnote{http://thoth.cc.e.ipl.pt/api/doc}.

\section{Organização da solução}
\subsection{Classes}
As classes que criámos foram maioritáriamente activities. O Eclipse ADT gera automáticamente as activities criadas com um modo de design (ficheiro XML)
localizado em /res/layout, e um ficheiro com extensão java localizado em /src/com/example/pdm\_serie1.

\subsection{Estrutura}
Utilizámos uma estrutura simples para este projecto:

\begin{description}
	\item[MainActivity] Mostra a lista das turmas escolhidas
		\begin{description}
			\item[SemestersActivity] Mostra a lista de semestres
			\item[SemesterClassesActivity] Mostra a lista de turmas do semestre escolhido
			\item[ContextMenu] Menu de contexto sobre a lista
			\begin{description}			
				\item[ClassInfo] Abre o browser na pagina inicial da turma selecionada
				\item[AssignmentActivity] Mostra os trabalhos dessa turma
				\item[ContextMenu] Menu de contexto sobre a lista
				\begin{description}											
					\item[AssignmentPage] Abre o browser na pagina do trabalho selecionado
					\item[AssignmentSchedule] Abre o calendário na edição de evento com as datas e titulos do trabalho selecionado
				\end{description}
			\end{description}
		\end{description}
\end{description}
					

\section{Componentes}

\subsection{Adapter}

David, explica aqui as alterações que fizeste em relação ao adapter sff

\subsection{Menu de contexto}
O menu de contexto é utilizado nas List Views que têm escolhas (como é o caso da list view de turmas e a list view de trabalhos). É acedido ao pressionar num item da respectiva list view durante um segundo.
Este componente para ser usado é primeiro necessário registar a view que o vai usar. Portanto no onCreate da Activity onde se encontra o menu é chamada a função registerForContextMenu(getListView()).
O menu de contexto pode ser criado através do xml ou de código no onCreateContextMenu. Nós ao adicionar cada item, adicionamos também a tag que é usada para identificar qual dos items foi selecionado na função
onCreateContextMenu. Nesta função recebemos o objecto item como parametro que tem a informação relativamente ao item que foi selecionado. Através deste objecto, identificamos o item pressionada e 
chamados a função correspondente.

\subsection{Calendário}
O calendário é usado para a marcação dos trabalhos. Através do menu de contexto presente na AssignmentActivity.class, é chamado o calendário criando um intent com os seguintes parametros:

\begin{lstlisting}
new Intent(Intent.ACTION_INSERT, "content://com.android.calendar/events").
intent.putExtra(Events.TITLE, "Titulo do evento");
intent.putExtra(CalendarContract.EXTRA_EVENT_BEGIN_TIME, "Data do inicio do evento");
intent.putExtra(CalendarContract.EXTRA_EVENT_END_TIME, "Data do fim do evento");
intent.putExtra(Events.DESCRIPTION, "Descritivo do evento");
\end{lstlisting}
Internamente, é chamada uma activity que processa o uri e utiliza os campos passados dentro do extra do intent para criar um evento no calendário do android.

\subsection{Shared Preferences}
É uma class que guarda informação que lhe é submetido. Esta informação mantêm-se guardada mesmo que a aplicação seja desligada. Usamos esta class para guardar  a lista de turmas que o utilizador selecionou
e também o semestre para evitar que seja preciso fazer uma query para recuperar a informação sempre ligue a aplicação.


\end{document}