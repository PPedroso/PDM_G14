\documentclass{article}
\RequirePackage[utf8]{inputenc}
\usepackage[portuguese]{babel}
\begin{document}

\title{Relatório do primeiro trabalho de PDM}
\author{David Raposo nº-----\and Pedro Pedroso nº32632 \and Ricardo Mata nº-----}

\maketitle

\section{Introdução}
O primeiro trabalho da disciplina de Programação de Dispositivos Móveis visa a criação de uma aplicação que, utilizando os conhecimentos básicos
adquiridos sobre a plataforma Android, permite que o utilizador possa ver alguma informação lectiva sobre as cadeiras do curso de Informática, com base em
informação extraída a partir da API do Thoth \footnote{http://thoth.cc.e.ipl.pt/api/doc}.

\section{Organização da solução}
\subsection{Classes}
As classes que criámos foram maioritáriamente activities. O Eclipse ADT gera automáticamente as activities criadas com um modo de design (ficheiro XML)
localizado em /res/layout, e um ficheiro com extensão java localizado em /src/com/example/pdm\_serie1.

\subsection{Estrutura}
Utilizamos uma estrutura simples para este projecto.

MainActivity(Mostra a lista das turmas escolhidas) 	
		SemestersActivity(Mostra a lista de semestres)
		SemesterClassesActivity(Mostra a lista de turmas do semestre escolhido)
		
		----ContextMenu-----
		ClassInfo(Abre o browser na pagina inicial da turma selecionada)
		AssignmentActivty
					----ContextMenu----
					AssignmentPage(Abre o browser na pagina do trabalho selecionado)
					AssignmentSchedule(Abre o calendário na edição de evento com as datas e titulos do trabalho selecionado)				
					

\section{Componentes}

\subsection{Adapter}

David, explica aqui as alterações que fizeste em relação ao adapter sff

\subsection{Menu de contexto}
O menu de contexto é utilizado nas List Views que têm escolhas (como é o caso da list view de turmas e a list view de trabalhos). É acedido ao pressionar num item da respectiva list view durante um segundo.
Este componente para ser usado é primeiro necessário registar a view que o vai usar. Portanto no onCreate da Activity onde se encontra o menu é chamada a função registerForContextMenu(getListView()).
O menu de contexto pode ser criado através do xml ou de código no onCreateContextMenu. Nós ao adicionar cada item, adicionamos também a tag que é usada para identificar qual dos items foi selecionado na função
onCreateContextMenu. Nesta função recebemos o objecto item como parametro que tem a informação relativamente ao item que foi selecionado. Através deste objecto, identificamos o item pressionada e 
chamados a função correspondente.

\subsection{Calendário}
O calendário é usado para a marcação dos trabalhos. Através do menu de contexto presente na AssignmentActivity.class, é chamado o calendário criando um intent com os seguintes parametros:

new Intent(Intent.ACTION_INSERT, "content://com.android.calendar/events").
intent.putExtra(Events.TITLE, "Titulo do evento");
intent.putExtra(CalendarContract.EXTRA_EVENT_BEGIN_TIME, "Data do inicio do evento");
intent.putExtra(CalendarContract.EXTRA_EVENT_END_TIME, "Data do fim do evento");
intent.putExtra(Events.DESCRIPTION, "Descrição do evento");

Internamente, é chamada uma activity que processa o uri e utiliza os campos passados dentro do extra do intent para criar um evento no calendário do android.

\subsection{Shared Preferences}
É uma class que guarda informação que lhe é submetido. Esta informação mantêm-se guardada mesmo que a aplicação seja desligada. Usamos esta class para guardar  a lista de turmas que o utilizador selecionou
e também o semestre para evitar que seja preciso fazer uma query para recuperar a informação sempre ligue a aplicação.

\section{Conclusion}
Write your conclusion here.

\end{document}