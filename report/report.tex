\documentclass{article}
\RequirePackage[utf8]{inputenc}
\usepackage[portuguese]{babel}
\begin{document}

\title{Relatório do primeiro trabalho de PDM}
\author{David Raposo \and Pedro Pedroso \and Ricardo Mata}

\maketitle

\section{Introdução}
O primeiro trabalho da disciplina de Programação de Dispositivos Móveis visa a criação de uma aplicação que, utilizando os conhecimentos básicos
adquiridos sobre a plataforma Android, permite que o utilizador possa ver alguma informação lectiva sobre as cadeiras do curso de Informática, com base em
informação extraída a partir da API do Thoth \footnote{http://thoth.cc.e.ipl.pt/api/doc}.

\section{Organização da solução}
\subsection{Classes}
As classes que criámos foram maioritáriamente activities. O Eclipse ADT gera automáticamente as activities criadas com um modo de design (ficheiro XML)
localizado em /res/layout, e um ficheiro com extensão java localizado em /src/com/example/pdm\_serie1.

\section{Conclusion}
Write your conclusion here.

\end{document}